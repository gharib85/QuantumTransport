%!TEX root = ../Main.tex
In this section, the basics of the tight-binding approximation for electron transport will be explained and demonstrated. The justification for using tight-binding in the proposed nanoporous graphene device simulations will also be laid out.
\subsection{Ballistic quantum transport}
As graphene is a two dimensional material that consists of carbon atoms arranged in a hexagonal pattern, features in such a material can approach nanometer and sub nanometer scales. Because of the small scale the electrical properties and the electrical nature of the material is greatly changed. Normal drift-diffusion current models describe electric charges per area and current per area, but because the conductor is graphene, it can be considered one dimensional. This makes drift-diffusion models insufficient to describe the electrical transport and properties of graphene because the drift-diffusion model is based on scattering of multiple electrons and the mean free path between scattering. Therefore it is common to use what is called a ballistic model when working with graphene. Because of the strong binding between atoms in graphene hardly any phonons are present in the material, even at room temperature. Using the ballistic model, this means that electrons are not affected by phonons, i.e. no electrons will be excited by phonons, and that the electrons move through the material as waves. Furthermore the model looks at only one electron at a time in the presence of an electron gas. This model has been used with big success for regular graphene and it seems the model also gives a good approximation for NPGs.
\subsection{\mathinhead{\pi}{\pi}-orbitals and \mathinhead{\pi}{\pi}-electrons}
\begin{wrapfigure}[8]{r}{.3\textwidth}
	\vspace{-2.3em}
	\centering
	\begin{tikzpicture}
		\chemfig{C*6(-C-C-C-C-C-)}
	\end{tikzpicture}
	\caption{Graphene lattices consists of hexagonal arrangements of carbon atoms.}\label{ring}
\end{wrapfigure}
The main scope of this paper is dealing with electron transport in novel nanoporous graphene devices.
When modeling such transport one needs to address the orbital structure of carbon lattices and later this will motivate the use of tight binding approximation and Green's functions. The two concepts of Tight Binding approximation and Green's functions will elaborated further in the coming sections.
In its basic form graphene can be divided into rings of carbon atoms as shown in \cref{ring}. In the (\(x,y\))-plane the carbon atoms are bound in \(sp^2\) orbitals as shown in \cref{sp2}.
\vspace{-2\baselineskip}
\begin{center}
\begin{figure}[H]
	\centering
	\resizebox{.4\textwidth}{!}{
		\begin{tikzpicture}
		    \node (x) at (-1,-3) {x};
		    \node (y) at (-2,-2) {y};
		    \draw[->] (-2,-3) -- (x);
		    \draw[->] (-2,-3) -- (y);
			\satom[name=C, color=blue, pos={(0,0)}]{
				blue/60/north east/2/1,
				blue/180/west/1,
				blue/300/south east/2/1
			}
			\satom[name=C, color=blue, pos={(1,1.4)}]{
				blue/0/east/2/1,
				blue/120/north west/1,
				blue/240/south west/2/1
			}
			\satom[name=C, color=blue, pos={(2.74,1.4)}]{
				blue/60/north east/1,
				blue/180/west/2/1,
				blue/300/south east/2/1
			}
			\satom[name=C, color=blue, pos={(3.74,0)}]{
				blue/0/east/1,
				blue/120/north west/2/1,
				blue/240/south west/2/1
			}
			\satom[name=C, color=blue, pos={(2.74,-1.4)}]{
				blue/60/north east/2/1,
				blue/180/west/2/1,
				blue/300/south east/1
			}
			\satom[name=C, color=blue, pos={(1,-1.4)}]{
				blue/0/east/2/1,
				blue/120/north west/2/1,
				blue/240/south west/1
			}
		\end{tikzpicture}}
	\caption{Carbon atoms in a hexagonal lattice are \(sp^2\) hybridised in the (\(x,y\))-plane.}\label{sp2}
\end{figure}
\end{center}
\vspace{-2\baselineskip}
This hybridisation lock all but one valence electron for the carbon atoms. These electrons exists in a p-orbital in the \(z\)-direction.
\cref{p} shows the valence orbitals of carbon.
\begin{figure}[H]
	\begin{center}
		\begin{tikzpicture}
			\orbital[pos = {(0,3)}] {s}
			\node[above] at (0,4) {s};
			\orbital[pos = {(2,3)}]{px}
			\node[above] at (2,4) {p$_x$};
			\orbital[pos = {(4,3)}]{py}
			\node[above] at (4,4) {p$_y$};
			\orbital[pos = {(6,3)}]{pz}
			\node[above] at (6,4) {p$_z$};
		\end{tikzpicture}
		\caption{The valence orbitals of carbon.}
		\label{p}
	\end{center}
\end{figure}
The last electron in the p\(_z\) orbital does not mix with the tightly bound s, p\(_x\) and p\(_y\) electrons and moves more freely. Thus these electrons have higher energies compared to the \(sp^2\) electrons and occupy states at the Fermi level. These electrons dominates transport in the graphene lattice. The p\(_z\) orbital is also known as the \(\pi\)-orbital and as such the electron lying there is called a \(\pi\)-electron. Through a carbon lattice the \(\pi\)-electrons will travel through \(\pi\)-orbitals. For a benzene ring the \(\pi\)-electrons at the highest occupied molecular state will travel through the p\(_\pi\)-orbitals switching sign as they travel as shown in \cref{sign}.
\begin{figure}[H]
	\begin{center}
		\pgfdeclarelayer{background}
		\pgfdeclarelayer{middle}
		\pgfdeclarelayer{foreground}
		\pgfsetlayers{background,middle,main,foreground}
		\begin{tikzpicture}
			\begin{pgfonlayer}{background}
				\orbital[pos = {(6,6)}]{-pz}
				\node[above] at (6,7) {-p$_\pi$};
				\orbital[pos = {(4,6)}]{pz}
				\node[above] at (4,7) {p$_\pi$};
				\draw[dashed, very thick] (6,6) -- (4,6);
				\draw[dashed, very thick] (7,4.73) -- (6,6);
				\draw[dashed, very thick] (4,6) -- (3,4.73);
			\end{pgfonlayer}
			\orbital[pos = {(7,4.73)}]{pz}
			\node[above] at (7,5.73) {p$_\pi$};
			\orbital[pos = {(3,4.73)}]{-pz}
			\node[above] at (3,5.73) {-p$_\pi$};
			\begin{pgfonlayer}{foreground}
				\orbital[pos = {(4,3.46)}]{pz}
				\node[above] at (4,4.46) {p$_\pi$};
				\orbital[pos = {(6,3.46)}]{-pz}
				\node[above] at (6,4.46) {-p$_\pi$};
				\draw[dashed, very thick] (4,3.46) -- (6,3.46);
			\end{pgfonlayer}
			\draw[dashed, very thick] (6,3.46) -- (7,4.73);
			\draw[dashed, very thick] (3,4.73) -- (4,3.46);
		\end{tikzpicture}
		\caption{When jumping from one carbon atom to another, the \(\pi\)-electron goes between p\(_\pi\)-orbitals. Such a jump is described by two matrix elements in the system's Hamiltonian.}
		\label{sign}
	\end{center}
\end{figure}
\subsection{Tight-binding}
Now that the transport carrying electrons are defined, one must choose a formalism for the transport itself. Introducing: \textbf{``The Tight-Binding approximation''}.
In this approximation the electrons are considered being tightly bound to the atoms. Contrary to a free electron gas approximation, the electrons does not spend time in between orbitals, but jump from orbital in atom \(a\) to orbital in atom \(b\). In this world view the Hamiltonian contains a matrix of hopping elements for a collection of neighbouring atomic orbitals, i.e. molecular orbitals, as well as the energy contained within each orbital (which will be addressed later on). This can be done by describing the orbitals as a Linear Combination of Atomic Orbitals (LCAO). The solution to the Schrödinger equation is then:
\begin{align}
	\Psi_{\mathrm{MO}} = \sum_{\alpha,R}c_{\alpha,R}\phi_{\alpha}(R)
\end{align}
where \(\phi_{\alpha}(R)\) is some atomic orbital at position \(R\), with \(\alpha\) denoting the valence of the orbital (\(2s,2p_x,2p_y,2p_z\)). In electron transport the states close to the Fermi level is of interest. These are namely the highest occupied molecular orbitals (HOMO), or the lowest unoccupied molecular orbitals (LUMO). As stated earlier only the \(\pi\)-electrons is then of interest.
The electrons' motion can be described with the hopping matrix of elements:
\begin{align}
	V_{pp\pi} = \bra{\phi_{\pi}(1)}\hat{H}\ket{\phi_{\pi}(2)}\label{V}
\end{align}
Physically this means that there is a potential between the \(\pi\) orbitals of neighbouring atoms \(1\) and \(2\). In our tight-binding approximation we consider only hop between nearest neighbours. The element
\begin{align}
	\epsilon_0 = \bra{\phi_{\pi}(1)}\hat{H}\ket{\phi_{\pi}(1)}
\end{align}
is the average energy of the electron on atom \(1\) and, it is normal to define the hopping energy relative to this:
\begin{align}
	\epsilon_0 = 0
\end{align}
If the atoms or their environment differs, so does the on-site potential.
\subsection{The benzene molecule}
\begin{wrapfigure}[7]{r}{.3\textwidth}
	\vspace{-2.3em}
	\centering
	\begin{tikzpicture}
		\chemfig{1*6(-2-3-4-5-6-)}
	\end{tikzpicture}
	\caption{Indices of a benzene molecule}\label{benz}
\end{wrapfigure}
As an example the Hamiltonian of benzene is considered. In \cref{benz} one can see the indices of a benzene molecule. Remember that \(\bra{\phi_{\pi}(1)}\hat{H}\ket{\phi_{\pi}(1)} = 0\) and \cref{V}, the Hamiltonian reads:
\begin{align}
	\mqty{                            \\ \\ \\ \vb{H} = V_{pp\pi}\\ \\ \\} \ \mqty{						&  \mqty{1 & 2 & 3 & 4 & 5 & 6} \\
		\mqty{1                           \\ 2 \\ 3 \\ 4 \\ 5 \\ 6} &	\mqty*(0 & 1 & 0 & 0 & 0 & 1 \\
	1 & 0 & 1 & 0 & 0 & 0             \\
	0 & 1 & 0 & 1 & 0 & 0             \\
	0 & 0 & 1 & 0 & 1 & 0             \\
	0 & 0 & 0 & 1 & 0 & 1             \\
	1 & 0 & 0 & 0 & 1 & 0)}\label{BH}
\end{align}
As a helping aid, \cref{BH} shows the atomic indices of the atom on the top and to the left of the matrix. This will give an understanding of how to work with such matrices.
The structure of the benzene molecule is rotationally symmetric and rotating the indices one sixth must yield the same Hamiltonian. Consider the energy eigenvector:
\begin{align}
	\phi = \mqty(c_1 & c_2 & c_3 & c_4 & c_5 & c_6)
\end{align}
There must exist an operator that rotates the indices as such:
\begin{align}
	C_6\phi = \mqty(c_2 & c_3 & c_4 & c_5 & c_6 & c_1)
\end{align}
The rotated Hamiltonian is the same, and thus \(C_6\) and \(\vb{H}\) commutes. The rotated vector must be an eigenvector with the same energy and it should be possible to find simultaneous eigenvectors to \(C_6\) and \(\vb{H}\).
\begin{align}
	C_6\phi = \mqty(c_2 & c_3 & c_4 & c_5 & c_6 & c_1) = \lambda\mqty(c_1 & c_2 & c_3 & c_4 & c_5 & c_6)
\end{align}
This operator \(C_6\) is represented with the matrix:
\begin{align}
	\vb{C}_6 = \mqty*(0 & 1 & 0 & 0 & 0 & 0  \\
	0                   & 0 & 1 & 0 & 0 & 0  \\
	0                   & 0 & 0 & 1 & 0 & 0  \\
	0                   & 0 & 0 & 0 & 1 & 0  \\
	0                   & 0 & 0 & 0 & 0 & 1  \\
	1                   & 0 & 0 & 0 & 0 & 0)
\end{align}
It can quickly be shown that the normalised eigenvectors to \(C_6\) are
\begin{align}
	\phi_n = \frac{1}{\sqrt{6}}\mqty(\lambda_n^0 & \lambda_n^1 & \lambda_n^2 & \lambda_n^3 & \lambda_n^4 & \lambda_n^5), \quad \lambda_n = \exp{-i2\pi n / 6}, \quad n = 0,1,2,3,4,5
\end{align}
These eigenvectors are also eigenvectors for \(\vb{H}\) with the eigenvalues:
\begin{align}
	\varepsilon_n = \lambda_n + \lambda_{n-1} = 2 \cos{n\pi/3}
\end{align}
Thus thanks to the rotational symmetry it was possible to find the eigenvectors and eigenenergies for the Hamiltonian.
