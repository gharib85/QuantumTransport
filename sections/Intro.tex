%!TEX root = ../Main.tex
\subsection{Balistic quantum transport}
\subsection{\mathinhead{\pi}{\pi}-orbitals and basis sets}
\begin{wrapfigure}[7]{r}{.3\textwidth}
	\vspace{-2.3em}
	\centering
	\begin{tikzpicture}
		\chemfig{C*6(-C-C-C-C-C-)}
	\end{tikzpicture}
	\caption{Graphene lattices consists of hexagonal arrangements of carbon atoms.}\label{ring}
\end{wrapfigure}
The main scope of this paper is dealing with electron transport in novel nanoporous graphene devices.
When modeling such transport one needs to adress the orbital structure of carbon lattices and later this will motivate the use of tightbinding and Green's functions.
In its basic form graphene can be devided into rings of carbon atoms as shown in \cref{ring}. In the (\(x,y\))-plane the carbon atoms are bound in \(sp^2\) orbitals as shown in \cref{sp2}.
\begin{figure}[H]
  \centering
  \resizebox{.4\textwidth}{!}{
		\begin{tikzpicture}
			\satom[name=C, color=blue, pos={(0,0)}]{
				blue/60/north east/2/1,
				blue/180/west/1,
				blue/300/south east/2/1
			}
			\satom[name=C, color=blue, pos={(1,1.4)}]{
				blue/0/east/2/1,
				blue/120/north west/1,
				blue/240/south west/2/1
			}
			\satom[name=C, color=blue, pos={(2.74,1.4)}]{
				blue/60/north east/1,
				blue/180/west/2/1,
				blue/300/south east/2/1
			}
			\satom[name=C, color=blue, pos={(3.74,0)}]{
				blue/0/east/1,
				blue/120/north west/2/1,
				blue/240/south west/2/1
			}
			\satom[name=C, color=blue, pos={(2.74,-1.4)}]{
				blue/60/north east/2/1,
				blue/180/west/2/1,
				blue/300/south east/1
			}
			\satom[name=C, color=blue, pos={(1,-1.4)}]{
				blue/0/east/2/1,
				blue/120/north west/2/1,
				blue/240/south west/1
			}
		\end{tikzpicture}}
		\caption{Carbon atoms in a hexagonal lattice are \(sp^2\) hybradised in the (\(x,y\))-plane.}\label{sp2}
\end{figure}
This hybradisation lock all but one valence electron for the carbon atoms. These electrons exists in a p-orbital in the \(z\)-direction.
\cref{p} shows the valence orbitals of carbon.
\begin{figure}[H]
	\begin{center}
		\begin{tikzpicture}
      \orbital[pos = {(0,3)}] {s}
      \node[above] at (0,4) {s};
			\orbital[pos = {(2,3)}]{px}
			\node[above] at (2,4) {p$_x$};
			\orbital[pos = {(4,3)}]{py}
			\node[above] at (4,4) {p$_y$};
			\orbital[pos = {(6,3)}]{pz}
			\node[above] at (6,4) {p$_z$};
		\end{tikzpicture}
		\caption{The valence orbitals of carbon.}
		\label{p}
	\end{center}
\end{figure}
The last electron in the p\(_z\) orbital does not mix with the tightly bound s, p\(_x\) and p\(_y\) electrons and moves more freely. The p\(_z\) orbital is also known as the \(\pi\)-orbital and as such the electron lying there is called a \(\pi\)-electron. Through a carbon lattice the \(\pi\)-electrons will travel through \(\pi\)-orbitals, switching sign as they go as shown in \cref{sign}.
\begin{figure}[H]
	\begin{center}
		\begin{tikzpicture}
			\orbital[pos = {(4,3)}]{pz}
			\node[above] at (4,4) {p$_\pi$};
			\orbital[pos = {(6,3)}]{-pz}
			\node[above] at (6,4) {-p$_\pi$};
		\end{tikzpicture}
		\caption{When going from one carbon atom to another, the \(\pi\)-electron goes betwenn p\(_\pi\) and -p\(_\pi\).}
		\label{sign}
	\end{center}
\end{figure}
\cite{calogero_electron_2019}
