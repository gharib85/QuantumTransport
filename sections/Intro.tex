%!TEX root = ../Main.tex
In recent years so called nano-porous graphene devices (NPGs) has been proposed for various applications. [Insert litt.] These devices are made up of single layered graphene with periodic holes (hence the porous) with which the intact graphene constitutes ribbons and bridges in the structures. Because of graphenes electrical properties [Insert litt], one should be able to finely control the electron currents in the devices and thus create nanometer circuits for use as e.g. chemical detectors. As a result of its novelty, the fabrication of such devices are limited. Before fabrication one must show promising effects through theorethical simulations.\newline
Employing Python and NumPy this project focusses on the development of nummerical routines by implementing the tight-binding approximation on NPGs and using Green's functions as well as clever recursion algorithms to simulate transmission and band structures.\newline
Generally speaking the community uses DFT-based simulations through tools like those from the SIESTA project (TBtrans), whith results analysed using SISL\cite{zerothi_sisl}. The DFT results can then be extrapolated to larger scales\cite{calogero_electron_2019}. However DFT programs run complex calculations and might seem as a blackbox for non-physcisists. To get a better understanding of tigh-binding and electron transport, we avoid DFT and rely solely on tight-binding simulations, whilst confirming the validity of the developed products by comparing result to those from SIESTA.\newline
The main scope is the development of the tight-binding scripts, comparing results with those of DFT calculations and discuss whether a clean tight-binding approach can sufficiently be used for the relatively simple NPGs.\newline
To summarise:\newline
\nth{1}: Apply quantum mechanics for electron transport in NPGs. \nth{2}: Use numerical methods (recursion algorithms, linear algebra) with NumPy to implement tight-binding. \nth{3}: Calculate band structures and transmission plots for various devices. \nth{4}: Gather single-particle Green’s functions and LDOS of said devices. \nth{5}: Compare the obtained results and discuss whether or not they sufficiently ressemble DFT based simulations.
The report is organised on the following way:
\begin{enumerate}
    \item \cref{theorysec,hamilsec,greensec,transec} deals with the development of our methodology. By introduction of basic theoretical concepts, followed by how these concepts are implemented practically through programming. 
    \item \cref{testsec} deals with the generated result on various NPGs and the comparison with DFT calculations with similar systems.
\end{enumerate}
The code repository (which also includes the \latex files for this report) can be found on Github: \faGithub \ \url{https://github.com/rwiuff/QuantumTransport}
