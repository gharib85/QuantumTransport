In order to calculate the transport of electrons in NPG one must obtain the Green's functions and self energies related to the unit cells of the system. In order to do so, a recursion algorithm must be implemented as calculations become too costly for the system, even if it is small. Especially the inversion of matrices required to obtain the Green's function can be very demanding computationally, when the system contains a lot of atoms. The recursion algorithm reduces the size of the system and thereby the amount of computational time required to obtain both the first cell Green's function, the Green's function within the chain (of repeated unit cells) sometimes called \(\mathbf{G}_{bulk}\) as well as the self energies related to those Green's functions. The recursion works by utilising that one can remove every second cell in an infinite chain of cells. As the chain originally was infinite, removing every second cell will just yield a new infinite chain. Every cell has with it, its hopping matrices and hamiltonian. The removal of every second cell is iterated, changing the effective interaction between the cells and thus the hopping matrices as well as the hamiltonians of the system. In the end the recursion algorithm produces re-normalised hamiltonians and hopping matrices, which can be used to obtain the Green's functions and self energies. 
\subsection{Obtaining first cell self energy and Green's function}

