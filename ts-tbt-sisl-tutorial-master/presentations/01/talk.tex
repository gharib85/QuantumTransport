
\input ../common.tex

\graphicspath{{fig/}}
\usepackage[export]{adjustbox}

\institute[2018, Nick R. Papior; DTU Nanotech]{\begin{tikzpicture}
      \node[shape=rectangle split,rectangle split parts=2,anchor=base] at (0,0)
      {DTU: sisl, TBtrans and TranSiesta workshop};
    \end{tikzpicture}}

\date{20. November 2018}
\title{Non-equilibrium Green function theory: 1 \& 2}
\author{Nick R. Papior}


\begin{document}

\begin{frame}
  \titlepage
\end{frame}

\begin{frame}
  \frametitle{Practical information}

  \small
  \begin{itemize}[<+->]
    
    \item Toilets: out to the left

    \item Danish tap water is safe to drink, don't waste money on buying expensive bottled
    water

    \item Start at 9 each day, sandwiches at 12 (no sandwiche Friday), cake at 14/15

    \item Poster session Wednesday at 17:30 in 341/342, dinner at 19:00

    \item Teaching assistants, Susanne and Gaetano

    \item Please ask questions during the workshop, you are here to learn!

    \item If there are things not covered in the tutorial, please ask and I will try and
    prepare a \emph{mini} tutorial for the last day

  \end{itemize}

  \uncover<+->{
      \begin{center}
        \large
        sisl is \emph{not} a static code! It evolves due to comments of users, so if you
        have a comment/suggestion/correction, please do so!
      \end{center}
  }
  
\end{frame}

\begin{frame}
  \frametitle{Outline}
  \tableofcontents
\end{frame}

\input introduction.tex
\input green.tex
\input se.tex
\input negf.tex
\input options.tex
\input electrodes.tex
\input device.tex

\end{document}
