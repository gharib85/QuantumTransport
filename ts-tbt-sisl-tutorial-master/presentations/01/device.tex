
\subsection{NEGF Algorithm in TBtrans}

\begin{frame}[fragile,label=algorithm]
  \frametitle{Input options}
  \framesubtitle{NEGF Algorithm in TBtrans}
  
  \begin{itemize}
    \item Reduce Green function calculation to region of interest ($D$)
    
    \item Algorithms determined from user-requested quantities
    
  \end{itemize}

  \vskip 4em

  \begin{itemize}
    \item Algorithm:
  \end{itemize}
  \begin{center}
    \incg[]{tbt-algorithm}
  \end{center}

  \begin{tikzpicture}[remember picture,overlay]
    \node at ($(current page.center)+(2.5,1.7)$) {%
        \incg[width=.7\linewidth]{inv-block3}%
    };
  \end{tikzpicture}

  \doicite{Papior \etal: \doi{10.1016/j.cpc.2016.09.022}}

  \hfill\hyperlink{electrode<6>}{\beamergotobutton{Return to electrode-setup}}

\end{frame}

\begin{frame}
  \frametitle{Input options}
  \framesubtitle{NEGF Algorithm in TBtrans}

  \begin{itemize}[<+->]
    \item Define $D$ via input
    
    \begin{tikzpicture}[thick,fixed node]
      
      % Define the electrode and we are done
      \begin{scope}
        \matrix {
            \node[fdf] {\%block TBT.Atoms.Device}; \\
            \node[fdf,ind] {atom [ 1 -- 3 ]}; \\
            \node[fdf,ind] {atom 5 11}; \\
            \node[fdf,ind] {atom -19}; \\
            \node[fdf] {\%endblock}; \\
        };
      \end{scope}
      \begin{scope}[xshift=8cm,yshift=1cm]
        \node[anchor=north,fill=black!70!white,text=white,align=center,text width=4.5cm] (A)
        at (0,0) {Always!\\\mbox{}\\\texttt{tbtrans -fdf TBT.Analyze}};
      \end{scope}
    \end{tikzpicture}
    
    \item TBtrans does everything else by it-self
    \item A smaller device region drastically reduces required memory(!), and increases
    throughput

    \item In non-orthogonal basis ($\SO\neq \mathbf I$) \fdf{TBT.Atoms.Device.Connect} may
    be useful

  \end{itemize}

  \incg[]{tbt-algorithm}

  \doicite{Papior \etal: \doi{10.1016/j.cpc.2016.09.022}}

\end{frame}

